% 2014
% Maciej Szeptuch
% IIUWr

\documentclass[12pt,leqno]{article}

\usepackage[utf8]{inputenc}
\usepackage{polski}
\usepackage{a4wide}
\usepackage[cm]{fullpage}

\usepackage{url}
\usepackage{graphicx}
\usepackage{epstopdf}
\usepackage{amsmath,amssymb}
\usepackage{bm}
\usepackage{amsthm}

%% Kropka po numerze paragrafu, podparagrafu, itp.
\makeatletter
    \renewcommand\@seccntformat[1]{\csname the#1\endcsname.\quad}
    \renewcommand\numberline[1]{#1.\hskip0.7em}
\makeatother

%%%%%%%%%%%%%%%%%%%%%%%%%%%%%%%%%%%%%%%%%%%%%%%%%%%%%%%%%%%%%%%%%%%%%%%%%%%%%%%%

\title{\LARGE \textbf{{Kurs: CUDA}}\\
      {\Large Projekt: Symulacja ruchu drogowego}\\
}
\author{Maciej Szeptuch}
\date{Wrocław, \today}

\begin{document}
\thispagestyle{empty}
\maketitle

%%%%%%%%%%%%%%%%%%%%%%%%%%%%%%%%%%%%%%%%%%%%%%%%%%%%%%%%%%%%%%%%%%%%%%%%%%%%%%%
%%%%%%%%%%%%%%%%%%%%%%%%% PLANOWANE %%%%%%%%%%%%%%%%%%%%%%%%%%%%%%%%%%%%%%%%%%%
%%%%%%%%%%%%%%%%%%%%%%%%%%%%%%%%%%%%%%%%%%%%%%%%%%%%%%%%%%%%%%%%%%%%%%%%%%%%%%%
\section{Planowane}
\subsection{Krótki opis}
Symulacja ruchu drogowego w czasie rzeczywistym z wykorzystaniem map
z OpenStreetMap. Dla określonego kawałka terenu będą losowane pozycje
startowe i końcowe dla określonej liczby samochodów po czym będzie można
w czasie rzeczywistym obserwować ich trasę. W miarę możliwości chciałbym
pozwolić na ustawianie rzeczy typu zbiór pozycji startowych/końcowych,
możliwość zmiany obszaru. Być może też jakieś przybliżanie/oddalanie mapy
(z tym że wtedy przydatna będzie też kompresja czasu)

\subsection{Podstawowy plan}
\begin{itemize}
    \item Obszar symulacji: okolice placu grunwaldzkiego we Wrocławiu
    \item Bardzo prosta fizyka ruchu samochodów - przyspieszanie do określonej
        maksymalnej prędkości na drodze, zachowywanie bezpiecznego odstępu od
        innych samochodów itd.
    \item Auta pojawiają się z określoną pozycją startową i końcową
    \item Parę prostych strategii ruchu (najkrótsza ścieżka, ścieżka
        reagująca na aktualne obciążenie dróg, itp.)

    \item Zamiast skrzyżowań: tunele i estakady(brak obsługi świateł)
\end{itemize}

\subsection{Możliwe rozszerzenia}
\begin{itemize}
    \item Możliwość zmiany obszaru symulacji(może wczytywanie na żywo
        z serwerów OSM, albo z pliku bazy)

    \item Możliwość wyboru punktów startowych i końcowych
    \item Wyłączanie poszczególnych dróg
    \item Kompresja czasu(przyspieszanie, spowalnianie)
    \item Obsługa świateł drogowych
\end{itemize}

%%%%%%%%%%%%%%%%%%%%%%%%%%%%%%%%%%%%%%%%%%%%%%%%%%%%%%%%%%%%%%%%%%%%%%%%%%%%%%%
%%%%%%%%%%%%%%%%%%%%%%%%%%%%%%% OSTATECZNA WERSJA %%%%%%%%%%%%%%%%%%%%%%%%%%%%%
%%%%%%%%%%%%%%%%%%%%%%%%%%%%%%%%%%%%%%%%%%%%%%%%%%%%%%%%%%%%%%%%%%%%%%%%%%%%%%%
\section{Ostateczna wersja}
\subsection{Opis}
Program pozwala na bardzo uproszczoną symulacje ruchu z użyciem map
z OpenStreetMap. Wykorzystując załączony importer można pliki osm.pbf (dostępne
np.\ z \url{http://metro.teczno.com/}) przetworzyć na pliki odczytywane przez
program. Pozwala to na symulacje z użyciem terenu dowolnego miasta, pod
warunkiem że zmieści się ono w pamięci. Samochody poruszają się losowo -
w momencie napotkania skrzyżowania próbują losowo znaleźć pierwszą wolną
krawędź, na którą mogą przejechać. W momencie w którym to się nie powiedzie
znajdują nowy punkt startowy i kontynuują losową wędrówkę. Można przybliżać
dowolny kawałek obszaru, a symulacja dalej kontynuuje na całej wczytanej mapie.

\subsection{(Nie)Zrealizowany plan}
\begin{itemize}
    \item Obszar symulacji: dowolny (w granicach rozsądku, możliwości sprzętu)
    \item Prosta fizyka samochodów: starają się zachowywać bezpieczną odległość,
        wszystkie mają takie same parametry - przyspieszenie itp.,
        wykorzystują dostępne z OSM dane do określenia aktualnej maksymalnej
        dozwolonej prędkości
    \item Auta pojawiają się w losowych miejscach i poruszają się po mapie
        w sposób losowy
    \item Domyślnie badanie kolizji na skrzyżowaniach jest wyłączone, aby je
        włączyć należy ustawić odpowiednie opcje kompilacji
    \item Interfejs jest praktycznie nieobecny: jedyne co można robić to
        poruszać mapą: przesuwać, obracać, przybliżać/oddalać
    \item Konfiguracja niestety sprowadza się do zmiany opcji w trakcie
        kompilacji
    \item W pewien sposób można manipulować mapami, zmieniając je za pomocą
        narzędzi dostępnych dla OSM po czym importując je do programu
\end{itemize}

%%%%%%%%%%%%%%%%%%%%%%%%%%%%%%%%%%%%%%%%%%%%%%%%%%%%%%%%%%%%%%%%%%%%%%%%%%%%%%%
%%%%%%%%%%%%%%%%%%%%%%%%% INSTRUKCJA OBSŁUGI %%%%%%%%%%%%%%%%%%%%%%%%%%%%%%%%%%
%%%%%%%%%%%%%%%%%%%%%%%%%%%%%%%%%%%%%%%%%%%%%%%%%%%%%%%%%%%%%%%%%%%%%%%%%%%%%%%
\section{Krótka instrukcja obsługi}
\subsection{Wymagania}
Program do działania wymaga \textbf{OpenGL}, \textbf{CUDA}, \textbf{QT5} oraz
\textbf{GLEW}. Do skompilowania programu importującego wymagany jest również
\textbf{google-protobuf}. Do samej kompilacji jest wymagany program \textbf{CMake}.

\subsection{Kompilacja}
Pomocne komendy(kompilacja, import mapy) zostały obudowane z użyciem Makefile
dla prostoty użycia. \\
Przed samą kompilacją można program nieco dostroić: \\
W pliku \textbf{CMakeLists.txt} w głównym katalogu znajdują się dwa makra
\_\_CUDA\_\_, \_\_CPU\_\_ odpowiadają one odpowiednio za obsługę symulacji
na karcie graficznej oraz na CPU. Można włączyć obydwie metody naraz wtedy
program będzie starał się równoważyć obciążenie pomiędzy GPU i CPU w celu
uzyskania jak największej wydajności. \\
W pliku \textbf{defines.h.in} w katalogu \textbf{src} znajdują się pozostałe
opcje konfiguracyjne, \\ można zmienić liczbę obsługiwanych rdzeni procesora
\textbf{CPU\_WORKERS} - w efekcie liczbę wątków próbujących rozłożyć
obciążenie pomiędzy wszystkie dostępne rdzenie; dokładność/szybkość
symulacji \textbf{SIMULATION\_FPS} domyślnie 24 kroki symulacji na sekundę;
liczba samochodów \textbf{CARS} jest domyślnie ustawiona na 131072;
można również eksperymentalnie uruchomić pełne badanie kolizji
\textbf{CROSS\_COLLISIONS} na skrzyżowaniach, niestety z powodu specyfiki
map OSM sprawia to, że bardzo szybko dochodzi do zakleszczenia przez co
większość samochodów przestaje jeździć tylko stoi na skrzyżowaniu bo nie
mają gdzie się ruszyć. \\
W tym pliku są także dostępne opcje dostosowywania CUDA, a mianowicie
liczba wątków \textbf{CUDA\_THREADS} oraz \textbf{CUDA\_CACHE} opcja
odpowiedzialna za wyłączenie używania pamięci shared - przydatna na
nowszych kartach (np.\ GTX 560 Ti) gdzie cache L2 zapewnia znacznie
większą wydajność niż gdy używamy pamięci shared. \\ \\

Sama kompilacja sprowadza się do wywołania komendy \textbf{make} lub
\textbf{make release} w głównym folderze, które odpowiednio skompilują
program w trybie debugowania lub w trybie maksymalnej wydajności.
Gdy komendy te się powiodą w katalogu \textbf{build} będą dostępne
binarki \textbf{trafficsim} oraz \textbf{trafficsim-import} (o ile
mamy google-protobuf) - odpowiednio symulator oraz program do importowania
danych z pliku osm.pbf. \\

\subsection{Uruchomienie}
Program \textbf{trafficsim-import} przyjmuje za argumenty plik osm.pbf
z danymi z OpenStreetMap oraz docelowe miejsce w którym ma zapisać
przetworzone dane. Dla wygody w pliku makefile zostały zdefiniowane
makra które pozwalają na przetworzenie plików wroclaw.osm.pbf,
poland.osm.pbf, tokyo.osm.pbf, planet.osm.pbf (które można pobrać
z już wspomnianej strony \url{http://metro.teczno.com/}) - odpowiednio
\textbf{make wroclaw}, \textbf{make poland}, \textbf{make tokyo},
\textbf{make planet} - o ile znajdują się one w podkatalogu \textbf{data}. \\ \\

Program \textbf{trafficsim} nie przyjmuje żadnych argumentów, zakłada
że w katalogu uruchomienia znajduję się plik \textbf{map.dat} z przetworzonymi
danymi mapy. Również dla wygody w Makefile zostało zdefiniowane makro
\textbf{make run}, które uruchamia go w katalogu data. Co pozwala na przykład
na wywołanie \textbf{make wroclaw}, \textbf{make run} co powinno uruchomić symulację
na mapie Wrocławia.

\end{document}
